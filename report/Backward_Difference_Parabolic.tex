\documentclass[a4paper]{article}
\usepackage[spanish,es-tabla]{babel}	% trabajar en español
\spanishsignitems	
%\usepackage{simplemargins}

%\usepackage[square]{natbib}
\usepackage{amsmath}
\usepackage{amsfonts}
\usepackage{amssymb}
\usepackage{bbold}
\usepackage{graphicx}
\usepackage{blindtext}
\usepackage{hyperref}
\usepackage{amsthm}
\newtheorem{theorem}{Teorema}
\newtheorem{lemma}{Lema}
\usepackage{algorithm}
%\usepackage{algorithmic}
\usepackage{algpseudocode}
%\usepackage{algorithm2e}
\usepackage{booktabs}

\setcounter{MaxMatrixCols}{20}

\begin{document}
\pagenumbering{arabic}

\Large
 \begin{center}
Método de Diferencias Finitas para ecuaciones parabólicas\\


\hspace{10pt}

% Author names and affiliations
\large
%Lic. Julio A. Medina$^1$ \\
Julio A. Medina\\
\hspace{10pt}
\small  
Universidad de San Carlos\\
Escuela de Ciencias Físicas y Matemáticas\\
Maestría en Física\\
\href{mailto:julioantonio.medina@gmail.com}{julioantonio.medina@gmail.com}\\

\end{center}

\hspace{10pt}

\normalsize
\section{Ecuaciones diferenciales parciales parabólicas}
La ecuación diferencial parcial parabólica o ecuación de calor también conocida como ecuación de difusión
\begin{equation}\label{eq::parabolic_partial_diff}
\frac{\partial u}{\partial t}(x,t) = \alpha^2 \frac{\partial^2 u}{\partial x^2}(x,t),\,\,\,\, 0<x<l, \,\,\,\, t>0.
\end{equation}
sujeta a las condiciones
\begin{equation}
u(0,t)=u(l,t)=0,\,\,\, t>0, \,\,\,\text{y}\,\,\, u(x,0)=f(x)
\end{equation}
El acercamiento para resolver la ecuación \ref{eq::parabolic_partial_diff} es el mismo utilizado en \cite{Medina} y \cite{Burden}. Es decir se define un retículo al seleccionar un $m>0$ y definir el paso $h=l/m$, despues escoger un paso temporal $k$, los puntos del reticulo en este caso son $(x_i, t_j)$, donde $x_i=ih$, para $i=0,1,\hdots,m$ y $t_j=jk$ para $j=1,2,\hdots$. 
\subsection{Método de diferencias atrasadas}



\begin{thebibliography}{99}
%% La bibliografía se ordena en orden alfabético respecto al apellido del 
%% autor o autor principal
%% cada entrada tiene su formatado dependiendo si es libro, artículo,
%% tesis, contenido en la web, etc


\bibitem{Burden} Richard L. Burden, J. Douglas Faires \textit{Numerical Analysis}, (Ninth Edition). Brooks/Cole, Cengage Learning. 978-0-538-73351-9

\bibitem{Medina} Julio Medina. \text{Método de Diferencias Finitas para ecuaciones elípticas}. \url{https://github.com/Julio-Medina/Finite_Difference_Method}

\bibitem{Varga} Richard S. Varga. \textit{Matrix Iterative Analysis}. Second Edition. Springer. DOI 10.1007/978-3-642-05156-2

%\bibitem{Feynman} 
%\bibitem{Hopfield} J.J. Hopfield. \textit{Neural Networks and physical systems with emergent collective computational abilities}. \url{https://doi.org/10.1073/pnas.79.8.2554}


%\bibitem{McCulloch} Warren S. McChulloch, Walter H. Pitts. \textit{A LOGICAL CALCULUS OF THE IDEAS IMMANENT IN NERVOUS ACTIVITY}. \url{http://www.cse.chalmers.se/~coquand/AUTOMATA/mcp.pdf}



\end{thebibliography}
\end{document}

